\documentclass[12pt,a4paper]{article}
\usepackage{xeCJK}
\setCJKmainfont{PMingLiU}[
    UprightFont = *,
    BoldFont = *,
    ItalicFont = *
]
\usepackage{geometry}
\usepackage{fancyhdr}
\usepackage{lastpage}
\usepackage{enumitem}
\usepackage{amsmath, amssymb}
\usepackage{multicol}
\usepackage{array}
\usepackage{longtable}
\usepackage{graphicx}
\usepackage{xcolor} %
\usepackage{titlesec}
% Page setup
\geometry{top=25mm,bottom=25mm,left=20mm,right=20mm}

% Header & footer
\pagestyle{fancy}
\fancyhf{}
% Define a custom gray color
\definecolor{mygray}{rgb}{0.5, 0.5, 0.5}

% Apply the color to the footer content
\fancyfoot[L]{\textcolor{mygray}{2025/2026-1st Term UT- S4 ICT}}
\fancyfoot[C]{\textcolor{mygray}{第 \thepage 頁}}
\renewcommand{\headrulewidth}{0pt}
\renewcommand{\footrulewidth}{0pt}
% Remove section numbering
\setcounter{secnumdepth}{0}


\titleformat{\section}{\normalsize\bfseries}{\thesection}{1em}{}
\titleformat{\subsection}{\normalsize\bfseries}{\thesubsection}{1em}{}
\titleformat{\subsubsection}{\normalsize\bfseries}{\thesubsubsection}{1em}{}

% Custom commands for underlines
\newcommand{\answerline}[1]{\underline{\hspace{#1}}}

\begin{document}

\begin{center}
{\Large\bfseries 佛教黃鳳翎中學}\\[0.5em]
{\Large\bfseries 2025/2026 上學期統測}\\[1em]
{\Large\bfseries 中四級 \quad 資訊及通訊科技 \quad 時限:60分鐘}\\[1.5em]
{\Large\bfseries 試題答題簿}
\end{center}

\vspace{1.5em}

班別:\answerline{3cm} \quad 班號:\answerline{2cm} \quad 考試日期:\answerline{3cm}\\[0.8em]
姓名:\answerline{8cm} \quad 考試時間:\answerline{3cm}\\[0.8em]

\textbf{本試卷必須用中文作答}

\vspace{1em}

\begin{center}
\begin{tabular}{|c|c|c|}
\hline
總 分 & 甲 部 & 乙 部 \\
\hline
& & \\
\hline
\end{tabular}
\end{center}

\vspace{1em}

\textbf{考生須知:}
\begin{enumerate}
\item 本卷分甲、乙兩部。
\item 甲部為多項選擇題;乙部為問答題。
\item 考生須在本試題答題簿第1頁之適當位置填寫考生姓名、班別及班號。
\item 本試卷滿分為50分。
\item \textbf{本試卷全部試題均須回答。}
\item 甲部的答案須填畫在多項選擇題的答題紙上,而乙部的答案則須寫在試題答題簿中預留的空位內。
\end{enumerate}

\newpage

\section{\textbf{甲部 多項選擇題(20分)}}

\textbf{本部共有20題。請選擇最合適的答案。}

\begin{enumerate}
\item 下列哪一項陳述有關信用卡號碼中的檢查數位是正確的?

\begin{enumerate}[label=\Alph*.]
\item 它可以偵測到信用卡號碼中的任何及所有錯誤。
\item 它可以確保信用卡號碼輸入正確。
\item 它可以更正信用卡號碼中的所有錯誤。
\item 它可以識別持卡人的銀行。
\end{enumerate}

\item 字元「W」和「Z」的十六進制 ASCII 代碼分別為 57 和 \answerline{2cm}

\begin{enumerate}[label=\Alph*.]
\item 3C
\item 5A
\item 59
\item 60
\end{enumerate}

\item 一個文本包含繁體中文和法文字符,應該使用下列哪一種字符編碼系統?

(1) UTF-8(統一碼)

(2) ASCII

(3) 大五碼

\begin{enumerate}[label=\Alph*.]
\item 只有 (1)
\item 只有 (3)
\item 只有 (1) 和 (3)
\item (1)、(2) 和 (3)
\end{enumerate}

\item 以下哪項 8 位元二進制補碼的數字,在計算過程中不會產生溢出誤差?

(1) 1010 1010 + 0101 0101

(2) 1000 0000 + 1000 0000

(3) 1111 1011 + 1101 1010

\begin{enumerate}[label=\Alph*.]
\item 只有 (1)
\item 只有 (1) 和 (2)
\item 只有 (1) 和 (3)
\item (1)、(2) 和 (3)
\end{enumerate}

\item 8 位元二進制補碼的範圍是多少?

\begin{enumerate}[label=\Alph*.]
\item $-2^7$ 至 $2^7$
\item $-2^7$ 至 $2^7 - 1$
\item $-2^7 - 1$ 至 $2^7 - 1$
\item $-2^7 + 1$ 至 $2^7 - 1$
\end{enumerate}

\item 對於聲音檔案而言,其 WAV 檔案的檔案大小大於其 MP3 檔案,因為

(1) WAV 音訊未經壓縮。

(2) WAV 比 MP3 檔案的時間較短。

(3) 可輕鬆編輯和處理 WAV 檔案,且不會造成品質損失。

\begin{enumerate}[label=\Alph*.]
\item 只有 (1)
\item 只有 (1) 和 (2)
\item 只有 (1) 和 (3)
\item (1)、(2) 和 (3)
\end{enumerate}

\item 以下哪些是圖像檔案格式?

(1) HEIC

(2) AI

(3) PNG

\begin{enumerate}[label=\Alph*.]
\item 只有 (1)
\item 只有 (3)
\item 只有 (1) 和 (3)
\item (1)、(2) 和 (3)
\end{enumerate}

\item 下列哪些工具最適合算法設計?

(1) 偽代碼

(2) 腦圖

(3) 流程圖

\begin{enumerate}[label=\Alph*.]
\item 只有 (1)
\item 只有 (1) 和 (3)
\item 只有 (1) 和 (2)
\item 只有 (2) 和 (3)
\end{enumerate}

\item 以下哪項陳述\textbf{不}正確?

\begin{enumerate}[label=\Alph*.]
\item 整數可以用於計算
\item 字符可用來儲存數字
\item 字串是一串字符
\item 浮點數(實數)可用來儲存英文字母
\end{enumerate}

\item 如果 X = 3 和 Y = 5,哪個布爾表達式會產生相同的結果?

(1) ((X < Y) OR (X + Y >= 10)) AND (X > 1)

(2) (X < Y) AND ((X + Y >= 10) OR (X > 1))

(3) ((X < Y) AND (X + Y >= 10)) OR (X > 1)

\begin{enumerate}[label=\Alph*.]
\item 只有 (1) 和 (2)
\item 只有 (2) 和 (3)
\item 只有 (1) 和 (3)
\item (1)、(2) 和 (3)
\end{enumerate}

\item 以下哪項可視為資訊?

\begin{enumerate}[label=\Alph*.]
\item 顧客名稱
\item 產品編號
\item 顯示去年銷售趨勢的圖表
\item 電話號碼
\end{enumerate}

\item 以下哪項學生資訊可以定義為布爾資料類型?

(1) 畢業狀況

(2) 居家地址

(3) 班別

\begin{enumerate}[label=\Alph*.]
\item 只有 (1)
\item 只有 (3)
\item 只有 (1) 和 (3)
\item 只有 (2) 和 (3)
\end{enumerate}

\item 下列哪項\textbf{不}是一種字符編碼系統?

\begin{enumerate}[label=\Alph*.]
\item 條碼
\item ASCII
\item 統一碼
\item 國標碼
\end{enumerate}

\item 餐廳收銀處通常使用下列哪項輸入/輸出設備?

(1) 磁條卡閱讀機

(2) 噴墨打印機

(3) 感熱式打印機

\begin{enumerate}[label=\Alph*.]
\item 只有 (1)
\item 只有 (3)
\item 只有 (1) 和 (2)
\item 只有 (1) 和 (3)
\end{enumerate}

\item 下列哪項有關噴墨打印機和點陣式打印機的比較是正確的?

(1) 噴墨打印機的輸出質素較點陣式打印機的為高。

(2) 噴墨打印機的打印速度較點陣式打印機的為低。

(3) 噴墨打印機在打印時的噪音較大。

\begin{enumerate}[label=\Alph*.]
\item 只有 (1)
\item 只有 (2)
\item 只有 (1) 和 (2)
\item 只有 (1) 和 (3)
\end{enumerate}

\item 下列哪項有關 RAM 和 ROM 的特徵是正確的?

\begin{center}
\begin{tabular}{cc}
\underline{RAM} & \underline{ROM} \\
\end{tabular}
\end{center}

(1) 易失性的 \quad 非易失性的

(2) 允許讀取和寫入 \quad 只允許讀取

(3) 可允許用戶升級 \quad 不允許用戶升級

\begin{enumerate}[label=\Alph*.]
\item 只有 (1)
\item 只有 (1) 和 (2)
\item 只有 (2) 和 (3)
\item (1)、(2) 和 (3)
\end{enumerate}

\item 下列哪項存貯設備是可覆寫的?

(1) 隨機存取記憶體

(2) 唯讀記憶體

(3) DVD-RW

\begin{enumerate}[label=\Alph*.]
\item 只有 (1)
\item 只有 (2)
\item 只有 (1) 和 (2)
\item 只有 (1) 和 (3)
\end{enumerate}

\item 小明的程式未能產生正確輸出,因此他修改了程式的流程圖。以下哪項解難步驟最能貼切地描述他的動作?

\begin{enumerate}[label=\Alph*.]
\item 分析問題
\item 設計算法
\item 開發程式
\item 測試及除錯
\end{enumerate}

\item 以下哪項是模組化的好處?

\begin{enumerate}[label=\Alph*.]
\item 減少程式的執行時間
\item 使程式不須測試
\item 使程式可重用
\item 使用戶更容易使用程式
\end{enumerate}

\item 以下哪項是下列算法的輸出?

A ← 5

B ← 10

C ← 15 - A * B

輸出 A + B - C

\begin{enumerate}[label=\Alph*.]
\item 20
\item 35
\item 50
\item 115
\end{enumerate}

\item 以下哪項是下列算法的輸出?

X ← 2

Y ← 3

Z ← 5

temp ← X

X ← Y

Y ← temp

Z ← Y - X

輸出 X - Y + Z

\begin{enumerate}[label=\Alph*.]
\item -2
\item 0
\item 2
\item 4
\end{enumerate}

\item 以下哪組數據類型和數據的組合是\textbf{不}正確的?

\begin{center}
\begin{tabular}{cc}
\underline{數據類型} & \underline{數據} \\
\end{tabular}
\end{center}

\begin{enumerate}[label=\Alph*.]
\item 整數 \quad 1
\item 布爾 \quad 1
\item 浮點數 \quad 1.0
\item 字符 \quad 1.0
\end{enumerate}

\item 下列算法中,當 X = 3,以下哪項是 Z 的值?

X ← 1

Z ← 7

當 X < 5

X ← X + 1

Z ← Z - X

\begin{enumerate}[label=\Alph*.]
\item 1
\item 2
\item 5
\item 7
\end{enumerate}

\item 大部分操作系統在更新最新版本時,都會提醒用戶盡快進行更新。以下哪項\textbf{不是}操作系統提醒用戶的原因?

\begin{enumerate}[label=\Alph*.]
\item 提供予舊版本操作系統的技術支援將會立刻停止
\item 堵塞安全性漏洞
\item 確保操作系統能與其他軟件兼容
\item 修復操作系統裡的缺陷
\end{enumerate}

\item 李老師準備將學生的考試成績輸入至學校的數據庫系統。在輸入數據前,李老師打算確認考卷在批改時沒有出錯。以下哪項最適合形容他的這個舉動?

\begin{enumerate}[label=\Alph*.]
\item 輸入數據兩次
\item 雙重數據輸入
\item 校對數據
\item 數據有效性檢驗
\end{enumerate}

\item 在某網上購物平台中,一個電郵地址只能用以註冊帳戶一次。以下哪項/些是適合使用於「電郵地址」欄的有效性檢驗?

(1) 格式檢查

(2) 一致性檢查

(3) 唯一性檢查

\begin{enumerate}[label=\Alph*.]
\item 只有 (1)
\item 只有 (2)
\item 只有 (1) 和 (3)
\item (1)、(2) 和 (3)
\end{enumerate}

\item 以下是某局部區域網路(LAN)的設計。X、Y 和 Z 分別是什麼?

% Note: Image would go here - placeholder for network diagram
[網路圖表位置]

\begin{center}
\begin{tabular}{ccc}
\underline{X} & \underline{Y} & \underline{Z} \\
\end{tabular}
\end{center}

\begin{enumerate}[label=\Alph*.]
\item 轉發器 \quad 交換器 \quad 防火牆
\item 網路接達點 \quad 交換器 \quad 數據機
\item 交換器 \quad 轉發器 \quad 防火牆
\item 網路接達點 \quad 防火牆 \quad 交換器
\end{enumerate}

\item 李老師正在為學校設置網路,以供網上考試之用。他選擇設立有線網絡而非無線網絡。以下哪項是他的主要考慮因素?

\begin{enumerate}[label=\Alph*.]
\item 有線網絡比無線網絡容易設置。
\item 就選擇設置地點來說,有線網絡比無線網絡更加靈活。
\item 有線網絡比無線網絡穩定。
\item 使用有線網絡比無線網絡容易管理使用者。
\end{enumerate}

\item 以下哪項是下列算法的輸出?

A ← 7

B ← 8

如果 (A + B <= 15) AND (A * B > 60) 則

A ← A * 3

輸出 A

否則

B ← B * 3

輸出 B

\begin{enumerate}[label=\Alph*.]
\item 7
\item 8
\item 21
\item 24
\end{enumerate}

\item 細閱以下算法的流程圖:

% Note: Flowchart image would go here - placeholder
[流程圖位置]

已知 A = 5、B = 4 和 C = 8,以下哪項是以上算法的輸出?

\begin{enumerate}[label=\Alph*.]
\item They can form a triangle.
\item They cannot form a triangle.
\item 沒有輸出
\item 出現錯誤
\end{enumerate}

\item A 是一個由索引 1 開始的陣列。以下哪項是下列算法的輸出?

A ← [3, 6, 9, 12, 15, 18, 21]

sum ← 0

設 i 由 2 至 6

sum ← sum + A[i]

輸出 sum

sum ← sum / 5

輸出 sum

\begin{enumerate}[label=\Alph*.]
\item 60
\item 60 12
\item 6 15 27 42 60 12
\item 9 18 30 45 63 12
\end{enumerate}

\item X 是一個由索引 1 開始的陣列。以下哪項是下列算法的目的?

X ← [1, 2, 3, 4, 5, 4, 3, 2, 1]

Y ← True

i ← 1

當 i < 9 AND Y = True

如果 X[i] > X[i+1] 則

Y ← False

i ← i + 1

輸出 i

\begin{enumerate}[label=\Alph*.]
\item 找出陣列停止由小至大排列的位置索引。
\item 找出陣列停止由大至小排列的位置索引。
\item 檢查陣列是否由小至大排列。
\item 檢查陣列是否由大至小排列。
\end{enumerate}

\end{enumerate}

\newpage

\section{\textbf{2. 問答題(30分)}}

\textbf{本試卷全部試題均須回答。請填寫最合適的答案。}

\begin{enumerate}

\item 美妮寫了一個算法,偽代碼如下所示:

\begin{center}
\begin{tabular}{|c|l|}
\hline
\textbf{行號} & \textbf{偽代碼} \\
\hline
1 & S ← 0 \\
\hline
2 & N ← 1 \\
\hline
3 & 執行 \\
\hline
4 & S ← S + N \\
\hline
5 & N ← N + 2 \\
\hline
6 & 當 N < 11 \\
\hline
7 & 輸出S, N \\
\hline
\end{tabular}
\end{center}

\begin{enumerate}[label=(\alph*)]
\item 芝芝閱讀了以上的規格後,發現沒有足夠的資訊來比較兩台設備的顯示質素。建議\textbf{兩項}芝芝所需有關設備的顯示器規格,讓她可進行比較。

\answerline{15cm}

\answerline{15cm} (2分)

\item 芝芝試用上述兩台設備後,發現即使它們的中央處理器和主記憶體的規格有明顯差別,兩者在顯示各種多媒體檔案的性能水平十分接近。為這情況建議一個可能的原因。

\answerline{15cm} (1分)

\item 平板電腦 Y 的用戶可享用由設備開發商提供的 5 GB 免費雲端存貯存服務。當用戶可使用互聯網連線時,他們可把平板電腦內的檔案和應用數據備份至雲端存貯平台上,以及從平台上把檔案和數據下載至平板電腦和桌上電腦。

\begin{enumerate}[label=(\roman*)]
\item 與microSD卡比較,舉出一個使用雲端存貯平台來貯存文件的\textbf{一個}優點和\textbf{一個}缺點。

\answerline{15cm}

\answerline{15cm}

\answerline{15cm} (2分)

\item 假設用戶可使用互聯網連線,試建議一種情況,使用雲端存貯服務的效率不及使用microSD卡。試扼要解釋。

\answerline{15cm}

\answerline{15cm} (2分)

\end{enumerate}

\item 芝芝最後購買了智能電話 X。她發現當透過 4G 網絡於該電話播放網上視像時,電池很快便會耗盡。建議\textbf{兩項}可使智能電話在播放網上視像時耗較少電能的設定。

\answerline{15cm}

\answerline{15cm} (2分)

\end{enumerate}

\item 以下算法用以遞增/遞減輸入值,直至兩者變為相同數值。

X ← 15

Y ← 1

重複

X ← X - 1

Y ← Y + 1

輸出 X, Y

直至 X = Y

\begin{enumerate}[label=(\alph*)]
\item 算法首三個輸出是什麼?

第一個輸出: X=\answerline{2cm} \quad Y=\answerline{2cm}

第二個輸出: X=\answerline{2cm} \quad Y=\answerline{2cm}

第三個輸出: X=\answerline{2cm} \quad Y=\answerline{2cm} (4分)

\item X 和 Y 被輸出多少次?它們的最終值是什麼?

\answerline{15cm}

\answerline{15cm} (3分)

\item 算法新增了一個變量 T,用以儲存 X 和 Y 的輸出次數。試完成算法。

\begin{tabular}{|p{13cm}|}
\hline
X ← 15 \\
\\
Y ← 1 \\
\\
T ← \answerline{3cm} \\
\\
重複 \\
\\
X ← X - 1 \\
\\
Y ← Y + 1 \\
\\
輸出 X, Y \\
\\
T ← \answerline{3cm} \\
\\
直至 X = Y \\
\\
輸出 T \\
\hline
\end{tabular}

(2分)

\item 試運用 while 循環改寫算法。

\begin{tabular}{|p{13cm}|}
\hline
X ← 15 \\
\\
Y ← 1 \\
\\
當 \answerline{3cm} \\
\\
X ← X - 1 \\
\\
Y ← Y + 1 \\
\\
輸出 X, Y \\
\hline
\end{tabular}

(1分)

\item 試舉出一組會令算法運行無限次的 X 和 Y 值。

\answerline{15cm}

\answerline{15cm} (2分)

\end{enumerate}

\end{enumerate}

\vspace{2em}

\begin{center}
\textbf{--- 試卷完 ---}
\end{center}

\end{document}
\item 執行程式後,S 和 N 的值是多少?

S: \answerline{2cm} \quad N: \answerline{2cm} \quad (2分)

\item 在第 2 行,如果 'N ← 1' 改為 'N ← 11',第 5 行會執行多少次?

\answerline{15cm} (1分)

\item 此程式的目的是什麼?

\answerline{15cm} (1分)

\end{enumerate}

\item 陣列 price 儲存了七份旅遊套票的價錢:

\begin{center}
\begin{tabular}{|c|c|c|c|c|c|c|c|}
\hline
\textbf{price} & 5000 & 5500 & 6000 & 7200 & 8100 & 8800 & 9200 \\
\hline
\textbf{索引} & 1 & 2 & 3 & 4 & 5 & 6 & 7 \\
\hline
\end{tabular}
\end{center}

試完成以下算法,檢查陣列 price 中的值是否由小至大排列。

\begin{center}
\begin{tabular}{|l|}
\hline
price ← [5000, 5500, 6000, 7200, 8100, 8800, 9200] \\
\\
sorted\_list ← \answerline{3cm} \\
\\
設 i 由 \answerline{3cm} \\
\\
如果 \answerline{5cm} 則 \\
\\
sorted\_list ← \answerline{3cm} \\
\\
如果 sorted\_list = True \\
\\
輸出'這是由小至大排列' \\
\\
否則 \\
\\
輸出'這不是由小至大排列' \\
\hline
\end{tabular}
\end{center}

(4分)

\item \begin{enumerate}[label=(\alph*)]
\item \begin{enumerate}[label=(\roman*)]
\item 試指出一個 PDF 相對於 DOCX 的優點。

\answerline{15cm} (1分)

\item 試指出一個 DOCX 相對於 TXT 的優點。

\answerline{15cm} (1分)

\end{enumerate}

\item \begin{enumerate}[label=(\roman*)]
\item 試指出一個 PNG 相對於 JPG 的優點。

\answerline{15cm} (1分)

\item 試指出一個 JPG 相對於 BMP 的優點。

\answerline{15cm} (1分)

\end{enumerate}

\item 如果只有 25 MB 儲存空間可用於儲存視訊,且每秒視訊需要200 KB,則最長的影片長度是多少?顯示你的計算步驟,並以分鐘為單位表達你的答案。

\answerline{15cm}

\answerline{15cm}

\answerline{15cm} (2分)

\end{enumerate}

\item 小美正在開發一個程式,下面的流程圖展示了程式的一部分:

% Note: Flowchart image would go here - placeholder
[流程圖位置]

\begin{enumerate}[label=(\alph*)]
\item 寫下以下輸入數的流程圖:

(1) A 的輸入值:11,B 的輸入值:7 \quad 輸出: \answerline{2cm}

(2) A 的輸入值:16,B 的輸入值:23 \quad 輸出: \answerline{2cm} (2分)

\item 流程圖中所述算法的目的為何?

\answerline{15cm}

\answerline{15cm} (2分)

\item 對上述需要輸入兩個值的算法,舉出一項可使用的數據有效性校驗。

\answerline{15cm}

\answerline{15cm} (1分)

\end{enumerate}

\item 李老師任教的學校嘗試推行混合教學模式。因此,李老師正在使用視像會議平台進行網上授課。

\begin{enumerate}[label=(\alph*)]
\item 試提出視像會議平台一個能夠協助教學的功能,並簡單解釋該功能如何協助教學。

\answerline{15cm} (2分)

\item 李老師希望在教室的佈告板張貼印有該網上課堂的連結,但學生卻表示要人手輸入連結十分困難。試提出一項改善建議。

\answerline{15cm}

\answerline{15cm} (1分)

\item 小明打算使用平板電腦參與網上會議,試舉出\textbf{兩項}平板電腦內常見的輸入設備。

\answerline{15cm}

\answerline{15cm} (2分)

\item 李老師打算傳送電子郵件向校長和其他老師匯報學生的學習進度。

\begin{enumerate}[label=\roman*.]
\item 其中一位學生發現李老師只在密件副本欄裏填寫了一個電郵地址。學生認為密件副本的唯一用途是隱藏此欄收件者的電郵地址,避免此欄的收件者互相得知對方的電郵地址,因此只在密件副本欄填寫一個電郵地址毫無作用。試解釋為何此學生的想法是錯誤的。

\answerline{15cm}

\answerline{15cm} (1分)

\item 李老師傳送該電郵時出現了警告,提示她電郵附件的檔案大小過大。試為李老師提供一種替代方式,以將檔案分享予收件者。

\answerline{15cm}

\answerline{15cm} (1分)

\end{enumerate}

\item 由於學校推行混合教學模式,學生許多時候都不會在教室上課。可是,學生離開教室時卻經常忘記關燈。試提出一種應用物聯網協助節約能源的方式。

\answerline{15cm}

\answerline{15cm} (1分)

\end{enumerate}

\item 芝芝想購買一台流動設備,用來在外出工作時間閱讀電子文件。她的秘書協助她對智能電話 X 和平板電腦 Y 作了一個簡單比較,並將兩台設備的部分規格如下列出。

\begin{center}
\begin{longtable}{|p{3cm}|p{5cm}|p{5cm}|}
\hline
& \textbf{智能電話 X} & \textbf{平板電腦 Y} \\
\hline
中央處理器 & 1.9 GHz 四核心 & 1.4 GHz 雙核心 \\
\hline
主記憶體 & 8 GB RAM & 2 GB RAM \\
\hline
存貯容量 & 256 GB 快閃記憶體 & 32 GB 快閃記憶體 \\
\hline
顯示 & 5" 多觸式 LED 背光屏幕 & 9.7"多觸式 LED 背光屏幕 \\
\hline
連通性 & IEEE 802.11 a/b/g/n、藍牙4.0及4G LTE & IEEE 802.11a/b/g/n、藍牙2.1 \\
\hline
重量 & 200 g & 650g \\
\hline
記憶卡插槽 & 1個適用於microSD卡插槽 & 沒有 \\
\hline
\end{longtable}
\end{center}

\begin{enumerate}[label=(\alph*)]