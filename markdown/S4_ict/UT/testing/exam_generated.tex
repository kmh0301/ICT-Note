% !TEX program = xelatex
\documentclass[12pt,a4paper]{article}

% --- Place all \usepackage commands in the preamble ---
\usepackage{xeCJK}
\setCJKmainfont{PMingLiU}[
    UprightFont = *,
    BoldFont = *,
    ItalicFont = *
]
\usepackage{geometry}
\usepackage{fancyhdr}
\usepackage{lastpage}
\usepackage{enumitem}
\usepackage{amsmath, amssymb}
\usepackage{multicol}
\usepackage{array}
\usepackage{longtable}
\usepackage{graphicx}
\usepackage{xcolor}
\usepackage{titlesec}
\usepackage{ifthen}
\usepackage{tikz} % Added for drawing answer lines/boxes

% Page setup
\geometry{top=25mm,bottom=25mm,left=20mm,right=20mm}

% Header & footer
\pagestyle{fancy}
\fancyhf{}
% Define a custom gray color
\definecolor{mygray}{rgb}{0.5, 0.5, 0.5}

% Apply the color to the footer content
\fancyfoot[L]{\textcolor{mygray}{2025/2026-1st Term UT- S4 ICT}}
\fancyfoot[C]{\textcolor{mygray}{第 \thepage 頁}}
\renewcommand{\headrulewidth}{0pt}
\renewcommand{\footrulewidth}{0pt}
% Remove section numbering
\setcounter{secnumdepth}{0}

\titleformat{\section}{\normalsize\bfseries}{\thesection}{1em}{}
\titleformat{\subsection}{\normalsize\bfseries}{\thesubsection}{1em}{}
\titleformat{\subsubsection}{\normalsize\bfseries}{\thesubsubsection}{1em}{}

% Custom commands for answer lines and boxes
\newcommand{\answerline}[1]{\par\vspace{0.5em}\noindent\rule{\linewidth}{0.4pt}\vspace{0.5em}}
\newcommand{\shortanswerline}[1]{\par\vspace{0.5em}\noindent\rule{#1}{0.4pt}\vspace{0.5em}}
\newcommand{\answerbox}[1]{\par\vspace{0.5em}\noindent\rule{\linewidth}{#1}\vspace{0.5em}}
\newenvironment{myquestionbox}[1]{%
    \par\vspace{1em}\hrule\vspace{1em}
    \textbf{答案框(共#1分)}\par
    \vspace{1em}\hrule\vspace{1em}
}{%
    \par\vspace{1em}\hrule\vspace{1em}
}

\begin{document}

\begin{center}
{\Large\bfseries 佛教黃鳳翎中學}\\[0.5em]
{\Large\bfseries 2025/2026 上學期統測}\\[1em]
\end{center}

\newpage

\section{\textbf{甲部 多項選擇題(20分)}}
\textbf{本部共有20題。請選擇最合適的答案。}

\begin{enumerate}
% --- PLACEHOLDER FOR MCQ QUESTIONS ---

    \item 以下哪一項硬件同時具備輸入和輸出的功能? (1 分)
    \begin{enumerate}[label=\Alph*.]
        \item 輕觸式屏幕 (Touch Screen)
        \item 繪圖板 (Graphics Tablet)
        \item 網絡攝影機 (Webcam)
        \item 揚聲器 (Speaker)
    \end{enumerate}
    
    \item 以下哪項關於數據 (Data) 和資訊 (Information) 的描述是正確的? (數據必定是數字,而資訊必定是文字。 分)
    \begin{enumerate}[label=\Alph*.]
        \item 一組學生的身高數字列表(例如:165
        \item  172
        \item  168)是資訊。
        \item 處理數據的目的是將其轉換為有意義的資訊。
    \end{enumerate}
    
    \item 某位網絡管理員的主要職責是甚麼? (1 分)
    \begin{enumerate}[label=\Alph*.]
        \item 設計和管理數據庫。
        \item 根據系統分析師的設計開發系統。
        \item 協助終端用戶使用資訊系統。
        \item 監控電腦系統之間的網絡通信。
    \end{enumerate}
    
    \item 在資訊處理的七個階段中,將數據上傳到雲端儲存服務屬於哪個階段? (1 分)
    \begin{enumerate}[label=\Alph*.]
        \item 收集 (Collection)
        \item 組織 (Organization)
        \item 傳輸 (Transmission)
        \item 分析 (Analysis)
    \end{enumerate}
    
    \item 某個網上登記系統要求用戶輸入的出生日期必須在「完成日期」之前。這應用了哪一種數據有效性檢驗? (1 分)
    \begin{enumerate}[label=\Alph*.]
        \item 類型檢查 (Type check)
        \item 範圍檢查 (Range check)
        \item 一致性檢查 (Consistency check)
        \item 唯一性檢查 (Uniqueness check)
    \end{enumerate}

    \newpage
    
    \item 在數據庫中,用於描述單一實體(例如一名特定學生)的所有相關屬性的集合稱為什麼? (1 分)
    \begin{enumerate}[label=\Alph*.]
        \item 欄 (Field)
        \item 記錄 (Record)
        \item 檔案 (File)
        \item 數據庫 (Database)
    \end{enumerate}
    
    \item 當收銀員掃描商品條碼時,系統發出嗶聲並顯示商品資訊。這個過程涉及以下哪些步驟? ((1)、(2) 和 (3) 分)
    \begin{enumerate}[label=\Alph*.]
        \item (1) 輸入
(2) 處理
(3) 輸出
        \item 只有 (1) 和 (2)
        \item 只有 (1) 和 (3)
        \item 只有 (2) 和 (3)
    \end{enumerate}
    
    \item 哪種儲存設備是以順序存取 (sequential access) 方式運作的? (1 分)
    \begin{enumerate}[label=\Alph*.]
        \item 固態硬碟 (SSD)
        \item 唯讀光碟 (CD-ROM)
        \item 磁帶 (Magnetic Tape)
        \item USB 快閃記憶體
    \end{enumerate}
    
    \item 一個8位元二進制代碼在奇偶檢測中傳送。若採用偶數檢測 (even parity),而數據位是 1011 0010,則包含奇偶檢驗位的完整代碼是甚麼? (1 分)
    \begin{enumerate}[label=\Alph*.]
        \item 1011 0010 0
        \item 1011 0010 1
        \item 0 1011 0010
        \item 1 1011 0010
    \end{enumerate}
    
    \item 以下哪個數字的值是最小的? (1 分)
    \begin{enumerate}[label=\Alph*.]
        \item 1101 1100$_{2}$
        \item 222$_{1}$$_{0}$
        \item DC$_{1}$$_{6}$
        \item 341$_{8}$
    \end{enumerate}
    
    \item 將十進制數 89 轉換為 8 位元二進制補碼 (two's complement) 是甚麼? (1 分)
    \begin{enumerate}[label=\Alph*.]
        \item 01011001
        \item 10100111
        \item 11011001
        \item 01011010
    \end{enumerate}
    
    \item 以下哪種關於向量圖 (vector image) 的描述是正確的? (1 分)
    \begin{enumerate}[label=\Alph*.]
        \item 它是由像素組成的。
        \item 放大後影像會變得模糊。
        \item 檔案大小通常比點陣圖大。
        \item 它使用數學公式來儲存影像。
    \end{enumerate}
    
    \item 一個需要包含繁體中文、日文和英文的文檔,最應使用哪種字符編碼系統? (1 分)
    \begin{enumerate}[label=\Alph*.]
        \item ASCII
        \item Big-5
        \item Unicode
        \item GB Code (國標碼)
    \end{enumerate}
    
    \item 一個由8位元組成的二進制數最多可以表示多少個不同的樣式? (1 分)
    \begin{enumerate}[label=\Alph*.]
        \item 8
        \item 16
        \item 128
        \item 256
    \end{enumerate}
    
    \item 某個需要8位數字的密碼,每個數位可以是 0-9 的數字或 A-F 的字母。這總共可以產生多少種不同的密碼組合? (1 分)
    \begin{enumerate}[label=\Alph*.]
        \item 10$^{8}$
        \item 16$^{8}$
        \item 8$^{1}$$^{0}$
        \item 8$^{1}$$^{6}$
    \end{enumerate}
    
    \item 將未壓縮的音訊檔案轉換為 MP3 格式,主要利用了以下哪種技術? (1 分)
    \begin{enumerate}[label=\Alph*.]
        \item 無損壓縮 (Lossless compression)
        \item 有損壓縮 (Lossy compression)
        \item 數位化 (Digitalisation)
        \item 加密 (Encryption)
    \end{enumerate}
    
    \item 以下關於條碼 (barcode) 和二維碼 (QR code) 的比較,哪一項是錯誤的? (1 分)
    \begin{enumerate}[label=\Alph*.]
        \item 二維碼通常能儲存比條碼更多的數據。
        \item 二維碼可以從任何方向掃描,而條碼通常只能從特定角度掃描。
        \item 條碼和二維碼都具備糾正錯誤的能力。
        \item 二維碼可以表示統一碼字符,而一些條碼只能表示數字。
    \end{enumerate}
    
    \item 一個 24 位元色深的圖像,其解像度為 800 x 600 像素。計算該圖像未壓縮的檔案大小 (以 MB 為單位)。 (1 分)
    \begin{enumerate}[label=\Alph*.]
        \item $\approx$ 1.37 MB
        \item $\approx$ 1.44 MB
        \item $\approx$ 11.52 MB
        \item $\approx$ 1.15 MB
    \end{enumerate}
    
    \item 在8位元二進制補碼系統中,進行 0110 0101 + 1100 1100 的運算。這會導致哪種錯誤? (1 分)
    \begin{enumerate}[label=\Alph*.]
        \item 轉錄錯誤 (Transcription error)
        \item 換位錯誤 (Transposition error)
        \item 溢位錯誤 (Overflow error)
        \item 奇偶檢驗錯誤 (Parity error)
    \end{enumerate}
    
    \item 小明使用掃描器將一份紙本文件轉換為數碼圖像,然後使用光符識別 (OCR) 軟件將圖像轉換為可編輯的文字檔案。這個過程涉及哪種轉換? (1 分)
    \begin{enumerate}[label=\Alph*.]
        \item 模擬數據轉換為模擬數據 (Analog to Analog)
        \item 模擬數據轉換為數碼數據 (Analog to Digital)
        \item 數碼數據轉換為模擬數據 (Digital to Analog)
        \item 數碼數據轉換為數碼數據 (Digital to Digital)
    \end{enumerate}
    
\end{enumerate}

\newpage

\section{\textbf{乙部 問答題(30分)}}
\textbf{本部共有5題。請在適當的答案框內作答。}
\begin{enumerate}
% --- PLACEHOLDER FOR Q&A QUESTIONS ---
\begin{enumerate}
\item 小麗正在為她的地理科專題報告搜集有關「城市熱島效應」的資料。 (6 分)
\begin{enumerate}[label=(\alph*)]
    \item 她可以透過哪兩種方法來收集原始數據? (2分)
    \item 在分析數據前,她需要先組織數據。試提出兩種整理數據的方法。 (2分)
    \item 小麗發現一篇網上文章的觀點非常偏頗。這涉及到資訊素養中的哪個重要範疇?試簡單解釋。 (2分)
\end{enumerate}
\item 一間學校正在設計一個新的學生資料數據庫。下表是其中一個資料表「Student」的結構:欄位名稱 (Field Name) 數據類型 (Data Type) 描述 StudentID 文字 (Text) 唯一的學生編號,格式為 SXXXXX ChineseName 文字 (Text) 中文姓名 EngName 文字 (Text) 英文姓名 BirthDate 日期/時間 (Date/Time) 出生日期 GradYear 整數 (Integer) 畢業年份 (6 分)
\begin{enumerate}[label=(\alph*)]
    \item 為「StudentID」欄位建議兩種合適的數據有效性檢驗方法。 (2分)
    \item 解釋為何「EngName」不適合被選為資料表的主鍵 (Primary Key)。 (2分)
    \item 學校希望加快按「GradYear」搜尋學生的速度。數據庫管理員應採取什麼措施?試解釋其原理。 (2分)
\end{enumerate}
\item 數值系統轉換與計算。 (7 分)
\begin{enumerate}[label=(\alph*)]
    \item 試將十六進制數 5E₁₆ 轉換為十進制數。 (2分)
    \item 試將十進制數 105₁₀ 轉換為二進制數。 (2分)
    \item 使用 8 位元二進制補碼計算 45₁₀ - 60₁₀。請列出計算步驟。 (3分)
\end{enumerate}
\item 小華正在製作一段關於校園生活的短片,並打算上傳至社交平台。他拍攝的影片屬性如下:* 解像度:1920 x 1080 * 幀率:30 幀/秒 * 色深:24 位元 * 長度:5 分鐘 * 音訊:已移除 (2 分)
\begin{enumerate}[label=(\alph*)]
    \item 為何大部分上傳到互聯網的影片(例如 YouTube)都會經過壓縮?試提出兩個原因。 (2分)
\end{enumerate}
\item 一間網上商店正設計其產品輸入介面。下圖顯示了產品數量輸入的介面,管理員輸入時將「20」誤輸入為「2O」(字母O)。 (2 分)
\begin{enumerate}[label=(\alph*)]
    \item 這屬於哪一種類型的數據輸入錯誤? (1分)
    \item 試建議一種數據有效性檢驗方法以防止此類錯誤。 (1分)
\end{enumerate}
\end{enumerate}

\end{enumerate}
\newpage

% Example for manual long question with answer space
% \item 某個問題,請在下面作答。(5分)
% \begin{myquestionbox}{5}
% \answerline{}
% \answerline{}
% \end{myquestionbox}

\end{document}