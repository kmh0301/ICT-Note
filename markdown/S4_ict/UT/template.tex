% =====================================================
% ICT考試卷LaTeX模板
% 適用於佛教黃鳳翎中學資訊及通訊科技科
% =====================================================

\documentclass[12pt,a4paper]{article}
\usepackage{xeCJK}
\setCJKmainfont{PMingLiU}[
    UprightFont = *,
    BoldFont = *,
    ItalicFont = *
]

% 確保字體大小與Word 12號字體一致
\usepackage{anyfontsize}
\renewcommand{\normalsize}{\fontsize{12pt}{14.4pt}\selectfont}
\renewcommand{\large}{\fontsize{14pt}{16.8pt}\selectfont}
\renewcommand{\Large}{\fontsize{16pt}{19.2pt}\selectfont}
\usepackage{geometry}
\usepackage{fancyhdr}
\usepackage{lastpage}
\usepackage{enumitem}
\usepackage{amsmath, amssymb}
\usepackage{multicol}
\usepackage{array}
\usepackage{longtable}
\usepackage{graphicx}

% Page setup
\geometry{top=25mm,bottom=25mm,left=20mm,right=20mm}

% Header & footer
\pagestyle{fancy}
\fancyhf{}
\fancyfoot[C]{第 \thepage 頁,共 \pageref{LastPage} 頁}

% Remove section numbering
\setcounter{secnumdepth}{0}

% Custom commands for underlines
\newcommand{\answerline}[1]{\underline{\hspace{#1}}}

% =====================================================
% 模板變數區 (AI生成題目時需要填入的內容)
% =====================================================

% 考試基本資訊
\newcommand{\examyear}{2025/2026}           % 學年
\newcommand{\semester}{上學期}              % 學期
\newcommand{\examtype}{統測}                % 考試類型
\newcommand{\gradelevel}{中四級}            % 年級
\newcommand{\subject}{資訊及通訊科技}       % 科目
\newcommand{\timelimit}{60分鐘}             % 時限
\newcommand{\examdate}{17/06/2025}          % 考試日期
\newcommand{\examtime}{8:30am -- 10:00am}   % 考試時間

% 分數分配
\newcommand{\totalscore}{50}               % 總分
\newcommand{\mcqscore}{32}                 % 選擇題分數
\newcommand{\essayscore}{18}               % 問答題分數
\newcommand{\mcqcount}{32}                 % 選擇題數量

\begin{document}

% =====================================================
% 考試卷頭部 - 使用標準12pt字體
% =====================================================
\begin{center}
{\fontsize{14pt}{16.8pt}\selectfont\bfseries 佛教黃鳳翎中學}\\[0.5em]
{\fontsize{14pt}{16.8pt}\selectfont\bfseries \examyear{} \semester{}\examtype{}}\\[1em]
{\fontsize{14pt}{16.8pt}\selectfont\bfseries \gradelevel{} \quad \subject{} \quad 時限:\timelimit{}}\\[1.5em]
{\fontsize{14pt}{16.8pt}\selectfont\bfseries 試題答題簿}
\end{center}

\vspace{1.5em}

% 確保所有正文使用12pt字體
\fontsize{12pt}{14.4pt}\selectfont

班別:\answerline{3cm} \quad 班號:\answerline{2cm} \quad 考試日期:\examdate{}\\[0.8em]
姓名:\answerline{8cm} \quad 考試時間:\examtime{}\\[0.8em]

\textbf{本試卷必須用中文作答}

\vspace{1em}

\begin{center}
\begin{tabular}{|c|c|c|}
\hline
總 分 & 甲 部 & 乙 部 \\
\hline
& & \\
\hline
\end{tabular}
\end{center}

\vspace{1em}

\textbf{考生須知:}
\begin{enumerate}
\item 本卷分甲、乙兩部。
\item 甲部為多項選擇題;乙部為問答題。
\item 考生須在本試題答題簿第1頁之適當位置填寫考生姓名、班別及班號。
\item 本試卷滿分為\totalscore{}分。
\item \textbf{本試卷全部試題均須回答。}
\item 甲部的答案須填畫在多項選擇題的答題紙上,而乙部的答案則須寫在試題答題簿中預留的空位內。
\end{enumerate}

\newpage

% =====================================================
% 甲部:多項選擇題 - 確保12pt字體
% =====================================================
\section{\textbf{\fontsize{12pt}{14.4pt}\selectfont 1. 多項選擇題(\mcqscore{}分)}}

\textbf{\fontsize{12pt}{14.4pt}\selectfont 本部共有\mcqcount{}題。請選擇最合適的答案。}

% 確保所有選擇題內容使用12pt字體
\fontsize{12pt}{14.4pt}\selectfont

% MCQ模板 - AI可以複製此結構來生成題目
% =====================================================
% 標準選擇題格式:
\begin{enumerate}

% 題目1 - 單選題範例
\item [題目文字]

\begin{enumerate}[label=\Alph*.]
\item [選項A]
\item [選項B]
\item [選項C]
\item [選項D]
\end{enumerate}

% 題目2 - 帶填空的選擇題範例
\item [題目文字] \answerline{2cm}

\begin{enumerate}[label=\Alph*.]
\item [選項A]
\item [選項B]
\item [選項C]
\item [選項D]
\end{enumerate}

% 題目3 - 多個子項目的選擇題範例
\item [題目文字]

(1) [子項目1]

(2) [子項目2]

(3) [子項目3]

\begin{enumerate}[label=\Alph*.]
\item 只有 (1)
\item 只有 (2)
\item 只有 (1) 和 (3)
\item (1)、(2) 和 (3)
\end{enumerate}

% 題目4 - 表格比較題範例
\item [題目文字]

\begin{center}
\begin{tabular}{cc}
\underline{項目A} & \underline{項目B} \\
\end{tabular}
\end{center}

(1) [比較項目1]

(2) [比較項目2]

(3) [比較項目3]

\begin{enumerate}[label=\Alph*.]
\item 只有 (1)
\item 只有 (1) 和 (2)
\item 只有 (2) 和 (3)
\item (1)、(2) 和 (3)
\end{enumerate}

% 題目5 - 算法/代碼題範例
\item [題目描述]

[偽代碼內容]

\begin{enumerate}[label=\Alph*.]
\item [選項A - 數值]
\item [選項B - 數值]
\item [選項C - 數值]
\item [選項D - 數值]
\end{enumerate}

% 以下為佔位符,AI生成時應替換為實際題目
% 題目6-32使用相同格式
\item [題目6內容...]
% ... 繼續到第32題

\end{enumerate}

\newpage

% =====================================================
% 乙部:問答題 - 確保12pt字體
% =====================================================
\section{\textbf{\fontsize{12pt}{14.4pt}\selectfont 2. 問答題(\essayscore{}分)}}

\textbf{\fontsize{12pt}{14.4pt}\selectfont 本試卷全部試題均須回答。請填寫最合適的答案。}

% 確保所有問答題內容使用12pt字體
\fontsize{12pt}{14.4pt}\selectfont

\begin{enumerate}

% 問答題模板 - AI可以參考此結構
% =====================================================

% 題目1 - 算法分析題範例
\item [題目描述]

\begin{center}
\begin{tabular}{|c|l|}
\hline
\textbf{行號} & \textbf{偽代碼} \\
\hline
1 & [代碼行1] \\
\hline
2 & [代碼行2] \\
\hline
3 & [代碼行3] \\
\hline
% ... 更多行
\end{tabular}
\end{center}

\begin{enumerate}[label=(\alph*)]
\item [子題目a]

[答案格式說明]: \answerline{2cm} \quad [更多答案位置]: \answerline{2cm} \quad ([分數]分)

\item [子題目b]

\answerline{15cm} ([分數]分)

\item [子題目c]

\answerline{15cm} ([分數]分)

\end{enumerate}

% 題目2 - 填空完成算法題範例
\item [題目描述]

\begin{center}
\begin{tabular}{|c|c|c|c|c|c|c|c|}
\hline
\textbf{[陣列名稱]} & [值1] & [值2] & [值3] & [值4] & [值5] & [值6] & [值7] \\
\hline
\textbf{索引} & 1 & 2 & 3 & 4 & 5 & 6 & 7 \\
\hline
\end{tabular}
\end{center}

[算法完成指示]

\begin{tabular}{|p{13cm}|}
\hline
[算法框架] \\
\\
[變數1] ← \answerline{3cm} \\
\\
設 i 由 \answerline{3cm} \\
\\
如果 \answerline{5cm} 則 \\
\\
[變數1] ← \answerline{3cm} \\
\\
[其餘算法邏輯] \\
\hline
\end{tabular}

([分數]分)

% 題目3 - 多部分問答題範例
\item \begin{enumerate}[label=(\alph*)]
\item \begin{enumerate}[label=(\roman*)]
\item [子題目i]

\answerline{15cm} ([分數]分)

\item [子題目ii]

\answerline{15cm} ([分數]分)

\end{enumerate}

\item [子題目b - 計算題]

[計算指示]

\answerline{15cm}

\answerline{15cm}

\answerline{15cm} ([分數]分)

\end{enumerate}

% 題目4 - 流程圖分析題範例
\item [題目描述]

% 流程圖位置佔位符
[流程圖位置]

\begin{enumerate}[label=(\alph*)]
\item [子題目 - 輸入輸出分析]

(1) [條件1]: 輸出: \answerline{2cm}

(2) [條件2]: 輸出: \answerline{2cm} ([分數]分)

\item [子題目 - 算法目的]

\answerline{15cm}

\answerline{15cm} ([分數]分)

\item [子題目 - 數據驗證]

\answerline{15cm}

\answerline{15cm} ([分數]分)

\end{enumerate}

% 題目5 - 應用情境題範例
\item [情境描述]

\begin{enumerate}[label=(\alph*)]
\item [功能分析子題]

\answerline{15cm} ([分數]分)

\item [改善建議子題]

\answerline{15cm}

\answerline{15cm} ([分數]分)

\item [技術規格子題]

\answerline{15cm}

\answerline{15cm} ([分數]分)

\item [多重子題]

\begin{enumerate}[label=\roman*.]
\item [複雜分析題]

\answerline{15cm}

\answerline{15cm} ([分數]分)

\item [替代方案題]

\answerline{15cm}

\answerline{15cm} ([分數]分)

\end{enumerate}

\item [IoT應用題]

\answerline{15cm}

\answerline{15cm} ([分數]分)

\end{enumerate}

% 題目6 - 設備比較題範例(含規格表)
\item [情境描述]

\begin{center}
\begin{longtable}{|p{3cm}|p{5cm}|p{5cm}|}
\hline
& \textbf{[設備A名稱]} & \textbf{[設備B名稱]} \\
\hline
[規格項目1] & [規格值A1] & [規格值B1] \\
\hline
[規格項目2] & [規格值A2] & [規格值B2] \\
\hline
[規格項目3] & [規格值A3] & [規格值B3] \\
\hline
[規格項目4] & [規格值A4] & [規格值B4] \\
\hline
[規格項目5] & [規格值A5] & [規格值B5] \\
\hline
[規格項目6] & [規格值A6] & [規格值B6] \\
\hline
[規格項目7] & [規格值A7] & [規格值B7] \\
\hline
\end{longtable}
\end{center}

\begin{enumerate}[label=(\alph*)]
\item [規格比較分析]

\answerline{15cm}

\answerline{15cm} ([分數]分)

\item [性能分析]

\answerline{15cm} ([分數]分)

\item [儲存比較題]

\begin{enumerate}[label=(\roman*)]
\item [優缺點比較]

\answerline{15cm}

\answerline{15cm}

\answerline{15cm} ([分數]分)

\item [效率比較情境]

\answerline{15cm}

\answerline{15cm} ([分數]分)

\end{enumerate}

\item [節能建議]

\answerline{15cm}

\answerline{15cm} ([分數]分)

\end{enumerate}

% 題目7 - 算法迭代分析題範例
\item [算法描述]

[偽代碼內容]

\begin{enumerate}[label=(\alph*)]
\item [輸出序列分析]

第一個輸出: [變數1]=\answerline{2cm} \quad [變數2]=\answerline{2cm}

第二個輸出: [變數1]=\answerline{2cm} \quad [變數2]=\answerline{2cm}

第三個輸出: [變數1]=\answerline{2cm} \quad [變數2]=\answerline{2cm} ([分數]分)

\item [執行次數和最終值分析]

\answerline{15cm}

\answerline{15cm} ([分數]分)

\item [算法完成 - 添加計數器]

\begin{tabular}{|p{13cm}|}
\hline
[原算法框架] \\
\\
[新變數] ← \answerline{3cm} \\
\\
[迭代結構] \\
\\
[計數邏輯] \\
\\
[變數更新] ← \answerline{3cm} \\
\\
[其餘算法] \\
\hline
\end{tabular}

([分數]分)

\item [循環結構轉換]

\begin{tabular}{|p{13cm}|}
\hline
[算法變數初始化] \\
\\
當 \answerline{3cm} \\
\\
[循環體內容] \\
\hline
\end{tabular}

([分數]分)

\item [邊界情況分析]

\answerline{15cm}

\answerline{15cm} ([分數]分)

\end{enumerate}

\end{enumerate}

\vspace{2em}

\begin{center}
\textbf{--- 試卷完 ---}
\end{center}

\end{document}

% =====================================================
% AI使用指南
% =====================================================
% 
% 1. 考試基本信息修改:
%    - 修改文檔頂部的 \newcommand 定義
%    - \examyear, \semester, \examtype, \gradelevel, \subject, \timelimit
%    - \examdate, \examtime, \totalscore, \mcqscore, \essayscore, \mcqcount
%
% 2. 字體大小確保:
%    - 所有正文內容強制使用12pt字體大小(相當於Word 12號字體)
%    - 標題使用14pt字體(相當於Word 14號字體)
%    - 使用 \fontsize{12pt}{14.4pt}\selectfont 確保字體大小一致性
%
% 3. 選擇題生成:
%    - 替換 [題目文字] 為實際題目內容
%    - 替換 [選項A], [選項B], [選項C], [選項D] 為實際選項
%    - 對於多子項題目,替換 [子項目1], [子項目2], [子項目3]
%    - 根據需要調整選項組合(如"只有 (1)"、"只有 (1) 和 (3)"等)
%
% 3. 問答題生成:
%    - 替換 [題目描述] 為實際題目內容
%    - 替換 [子題目a], [子題目b] 等為實際子題目
%    - 替換 [分數] 為實際分數分配
%    - 根據需要添加或刪除子題目
%    - 對於表格題,替換表格內容
%    - 對於算法題,替換偽代碼內容
%
% 4. 特殊格式:
%    - 流程圖:使用 [流程圖位置] 佔位符,需要插入實際圖片
%    - 網路圖:使用相同的圖片佔位符
%    - 算法框:使用 tabular 環境保持格式
%    - 長答案行:使用 \answerline{15cm}
%    - 短答案行:使用 \answerline{2cm} 或 \answerline{3cm}
%
% 5. 常用ICT主題:
%    - 數據表示和編碼
%    - 算法和程序設計
%    - 硬件和軟件
%    - 網絡和通訊
%    - 數據管理和數據庫
%    - 資訊系統和應用
%    - 多媒體和文件格式
%    - 資訊安全和私隱
%    - 新興科技(IoT、AI、雲計算等)